\documentclass[../main.tex]{subfiles}

\begin{document}

\section{Auditing Management}
\par The idea of having an auditing file is to record the purpose of each critical action of the authorised users (reading or writing a file). The challenge is that each record (each purpose) of an auditing file must be protected and be self-standing (can be encrypted and decrypted without any other information than those present inside the record).
\par Each auditing file is linked to a single file in the filesystem and is composed of multiple independent records.
\par In order to solve these challenges, we simply consider an audit entry to be like a metadata (cfr. Section \ref{section:lauxus:metadata}). In this context, the metadata secured section is composed of: the purpose of the action, the time and the user who initiated the action. Each record are then chained requiring no additional encryption. Indeed, the cryptographic context of each record (metadata) is secured thanks to the audit root key.


\end{document}