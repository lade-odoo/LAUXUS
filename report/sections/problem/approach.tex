\documentclass[../main.tex]{subfiles}

\begin{document}

\section{Approach overwiew}
\label{section:problem:approach}
% not forget root key
% coordination on key creation between admin and aud itor
% 
\begin{figure}[h]
    \centering
    \includegraphics[width=\textwidth]{../../images/problem/approach}
    
    \label{figure:problem:approach}
    \caption{Overview of the protocol to share the FileSystem}
\end{figure}

\par After all the previous introduction sections, we can start to explain the solution we designed to address the problem. In this section, we will explain the solution at a high level. Detailed insights will be discussed in the Chapter \ref{chapter:lauxus}. The Figure \ref{figure:problem:approach} presents this protocol and will be the heart of this section.
\par As presented in the use case Section \ref{section:problem:use_case} the protocol happens between multiple parties. By comparing with the Figure \ref{figure:problem:use_case}, we see that we have one more actor here: the administrator (Alice). We didn't talked about her in the previous section as it was not necessary for understanding the goals and the general idea of the problem. The role of the administrator is to create the filesystem and upload it to the cloud storage. She is responsible for the user management: registering and assigning specific roles to each user. Some readers might be confused between the difference between the administrator and the owner user: the administrator manages the users while the owner user manages which user access which file, we named this the data management.\\

\par Next, we will take a look at the core of the filesystem, its content. In order to protect confidential information against attacker, the administrator will encrypt the filesystem with a secret key (the root key, in red). Then she will upload the encrypted filesystem to the remote storage. In order to avoid sharing the root key in clear, she will only share it after encrypting it with another key (the blue key). This last key will only be transmitted to authorized users through a secure out-of-band channel. This is where all the power of SGX comes into action: being able to manipulate a secret on a client's computer without revealing any information on this secret to the client itself. Here, the secret we are talking about is the root key. In this way, the intended user, for example Carl, can decrypt the key located on the remote storage and then decrypt the filesystem to access it, without knowing anything about the root key. This means that every party must have an SGX-enabled CPU on their computer. The blue key, once shared, will be securely stored on each computer by leveraging SGX sealing capabilities.\\

\par Last, we have the auditing files. It works in the same manner as with the filesystem content and the administrator. The equivalent of the administrator is the auditor who owns an audit root key (the purple key). The sharing of the audit root key follows the exact same procedure as with the root key. As explained above, the audit root key must also be shared in order to have a working filesystem. We are using two different keys in order to have segregation of duties. The idea being that an organization, here the hospital, can't at the same time read in plaintext the filesystem content and the auditing files. We don't want an organization to change the auditing file at their advantage. As audit files are used for GDPR purposes, only a trusted entity should read those, thus owning the audit root key. There is a 1-to-1 mapping between the auditing files and the files in the filesystem. In practice, the filesystem will create one entry in the appropriate audit file on each read or write of any user, no matter its role.

\par In the end, the security of the filesystem is reduced to the security of the two root keys (the root key and the audit root key).
\par An UML state diagram can be found in Appendix \ref{appendix:state_diagram} for more detailed information.


\end{document} 