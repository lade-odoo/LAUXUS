\documentclass[../main.tex]{subfiles}

\begin{document}

\section{\textit{FUSE}}
\label{section:theoric:fuse}

\par \textit{FUSE} stands for Filesystem in User Space. It consists of a Kernel module and a user-space library. It is uses in order to override the classical comportment of the UNIX Filesystem. It works by forwarding the system calls from the kernel into the user-space library for custom behaviour. Many Filesystem have already been implemented as can be seen in \cite{fuseExamples} (e.g: turning all the filenames to lower case, etc). A great advantage of using \textit{FUSE} is that they can be used on top of each other (e.g: combine a filesystem that turns the path in lower case and one that create a directory when the filename is composed of a "-"). Another often favoured advantage is that it is very easy for a lambda developer to create a custom Filesystem thanks to its user-friendly user-space library.
\par Lastly, we can note that there are two version of user-space library: either a low level one or a high level one. The main difference is that the first one works with inode whereas the latter one works with filenames.


\end{document}